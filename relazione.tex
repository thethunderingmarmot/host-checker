\documentclass[a4paper]{report}
\usepackage{graphicx} % Required for inserting images
\usepackage[italian]{babel}
\usepackage{hyperref}
\usepackage{xcolor}
\usepackage[dvipsnames]{xcolor}

\title{Relazione sul progetto "Host Checker" \\ Traccia 3 - Monitoraggio di Rete}
\author{Tommaso De Tommaso}
\date{2 luglio 2024}

\begin{document}

\maketitle

\chapter{Introduzione}

\section{Consegna}
Realizzare uno script Python per monitorare lo stato di una rete, controllando la disponibilità di uno o più host tramite il protocollo ICMP (ping). Lo script deve consentire all'utente di specificare gli indirizzi IP degli host da monitorare e deve visualizzare lo stato (online/offline) di ciascun host.

\section{Soluzione}
L'applicazione assolve la consegna utilizzando i socket per inviare pacchetti ICMP ed il multithreading per farlo contemporaneamente a più host.

\chapter{Documentazione}
\section{Funzionamento di base}
\begin{figure}[h!]
\begin{minipage}{.5\textwidth}
    \centering
    \includegraphics[width=0.5\linewidth]{image.png}
\end{minipage}
\begin{minipage}{.5\textwidth}
    \centering
    \includegraphics[width=0.5\linewidth]{image2.png}
\end{minipage}
\end{figure}
L'applicazione si presenta come in figura: una finestra che contiente una casella di testo, un pulsante ed una zona (all'inizio vuota) dove appariranno i vari host da controllare.

Ogni host aggiunto verrà visualizzato come una riga di testo colorata che ne indicherà l'hostname ed il suo stato.

La colorazione delle righe di testo accompagna la dicitura fornendo un feedback più immediato: 
\begin{itemize}
    \item \colorbox{TealBlue}{\textbf{blu} indica l'attesa di una risposta}
    \item \colorbox{Green}{\textbf{verde} indica che l'host è online}
    \item \colorbox{Gray}{\textbf{grigio} indica che l'host è offline}
    \item \colorbox{Red}{\textbf{rosso} indica il verificarsi di un problema}
\end{itemize}

\subsection{Istruzioni}
E' necessario aver installato \href{https://www.python.org/}{Python} per proseguire. Una volta eseguito lo script, l'aggiunta di un host è molto semplice, basterà scrivere il suo hostname nella apposita casella di testo e poi premere Invio sulla tastiera, oppure sul pulsante "Aggiungi".

\newpage
\section{Funzionamento avanzato}
Lo script usa solo librerie già incluse in Python: l'interfaccia grafica è stata realizzata con la libreria \textit{tkinter}, sono stati utilizzati i socket forniti dall'omonima libreria \textit{socket}, la costruzione del pacchetto ICMP avviene utilizzando strumenti dalla libreria \textit{struct}, mentre il multithreading è reso possibile dalla libreria \textit{threading}.

Quello che avviene dietro le quinte è che ad ogni pressione di Invio, o del pulsante "Aggiungi", viene creato un thread che invia un pacchetto ICMP all'host, poi aspetta massimo 5 secondi per una risposta e, se la riceve, l'host viene contrassegnato come online; in caso contrario, l'host viene contrassegnato come offline. A questo punto il thread rimane in attesa per altri 5 secondi prima di ripetere tutto, finché l'applicazione non viene chiusa.

Infatti, se si dovesse aggiungere un host che è online al momento dell'inserimento, ma che successivamente diventa offline, si potrà notare come, entro massimo una decina di secondi, l'applicazione ne aggiornerà lo stato senza problemi.

\subsection{Dettagli}
Durante la costruzione del pacchetto ICMP è necessario includere il suo checksum nel pacchetto stesso, questo viene calcolato con una piccola, ma complessa funzione denominata \textit{get\_checksum} che segue le direttive specificate in \href{https://datatracker.ietf.org/doc/html/rfc1071}{RFC1071}.

\end{document}
